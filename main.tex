\documentclass[12pt]{article}
\usepackage[utf8]{inputenc}

\usepackage{algorithm}% http://ctan.org/pkg/algorithms
\usepackage{algpseudocode}% http://ctan.org/pkg/algorithmicx
\usepackage{graphicx}
\usepackage{fancyhdr}
\usepackage{setspace}
\usepackage{minted}
\usepackage{biblatex}
\usepackage{svg}
\usepackage{listings}
%inline verbatim%
\usepackage{fancyvrb}

\addbibresource{references.bib}

\pagestyle{fancy}

% remove header
\renewcommand{\headrulewidth}{0pt}
\fancyhead{}

% set margin
\usepackage[
  top=1in,
  bottom=1in,
  left=1.5in,
  right=1in
]{geometry} 

\begin{document}
\begin{titlepage}
   \begin{center}
        Kalamazoo College\\
        Senior Individualized Project in Computer Science
       \vspace*{3cm}
 
       \textbf{Developing a Reusable Framework for Data Fetching and Mutation}
 
       \vspace{0.5cm}
       GraphQL to SQL query mapping using existing Typescript technologies
 
       \vspace{1.5cm}
 
       \textbf{Joshua Gibson}
 
       \vspace{4cm}
 
       Faculty Supervisor:\\
       Dr. Eric Barth\\
       Assistant Provost and Professor of Mathematics\\
       Kalamazoo College\\
       Kalamazoo, MI
       
       \vspace{3cm}
       
       A paper submitted in partial fulfillment of the requirements for the degree of Bachelor of Arts at Kalamazoo College
       
       \vspace{2cm}
       2019
 
   \end{center}
\end{titlepage}

\pagenumbering{gobble}

%empty page
\null \newpage


\pagenumbering{roman}
\setcounter{page}{2}

\doublespacing

\addcontentsline{toc}{section}{Acknowledgements}
\section*{Acknowledgements}
This project would not have been possible without the help of dozens of people, both during the process of the SIP and before it was even a thought.  While this is by no means an exhaustive list, there are a number of people who must be thanked by name for their help.  I of course have to start by thanking my advisor Dr. Eric Barth for taking me on as an advisee.  This project has morphed a lot since he agreed to advise me on it, but he has been fully supportive in letting the project evolve naturally.  His guidance and support through these changes has been crucial.  In helping plan my initial project for my SIP, I must also thank Professor Andrew Koehler for his similar willingness to advise a project somewhat outside of his comfort zone.

As I developed the technologies discussed in my SIP over the summer, I would not have been able to do so without the mentorship and guidance of my co-workers at Maestro.  A huge thanks goes out to John Pinkster, Ben Meden, and Tyler VanderMaas for teaching me so much about back-end development.  Additionally, I have to thank Baxter Banghart for not entertaining phone calls throughout the summer where I would excitedly plan and develop these crazy ideas about GraphQL, but also for helping push me to be a better over the last eight years we've collaborated together.

This paper itself would also not be possible without the help of many people.  Thanks to Dr. Alyce Brady for entertaining my questions throughout the SIP Seminar and offering her years of wisdom she's gained through advising CS SIPs.  Thanks to all of my fellow CS students who proofread my paper as it was in the process of being written.  Additionally, thanks to my sister Amanda Gibson, for proofreading the paper as it neared its final draft.

Finally, I have to thank those who helped me get to where I am today.  A huge thanks goes out to Dr. Kelly Schultz.  She was the teacher who inspired my passion for Computer Science at such a young age and lead me to the computer science department at Kalamazoo College.  It makes me happy to know she's still teaching and inspiring young students today.  My other CS educators at Kalamazoo College, Dr. Alyce Brady, Dr. Pam Cutter, and Dr. Gerry Howser, have also played a huge part in developing me a computer scientist.  Without their education, I would not have the skills or ambition to tackle this kind of project.   And of course, I must also thank my parents, Scott and Kristine Gibson.  They have been nothing but supportive throughout all my education, whether it was helping me dual enroll in classes to help fuel my passion for computer science, encouraging me to study music in many different contexts and venues, or supporting me throughout college. No aspect of this project would be possible without them.
\newpage
\newpage

\section*{Abstract}
\addcontentsline{toc}{section}{Abstract}
This project seeks to develop a framework to minimize the amount of code required from developers to enable data fetching and mutation on a relational database.  Instead of manually generating SQL queries to interact with the database, GraphQL queries are translated to SQL by the framework.  Using technologies such as GraphQL, SQL, Nest.js, and Type GraphQL, a set of reusable classes and functions are proposed that should ultimately reduce the amount of repetitive code in APIs developed for web-enabled applications. This project was developed within the context of a personal project, Practice Liszt, and throughout an internship and part time position at Maestro LLC in Kalamazoo, Michigan.  The framework proposed by the project is limited in its current GraphQL parsing ability but shows strong potential to integrate with these existing technologies to enable developers to extend the base creating, reading, updating, and deleting functionality.  In addition to the future improvements in GraphQL parsing in this project, other future framework features such as sorting, pagination, and filtering are discussed.
\newpage
\singlespacing
\tableofcontents
\doublespacing
\newpage

\listoffigures
\newpage


\pagenumbering{arabic}
\section{Introduction}

Programming is so often a balance of writing original functionality while abstracting away often repeated code.  This balance is a delicate game.  Leave the developer with too much control and the lack of abstraction makes implementing new features difficult.  Every little bit of functionality has to be re-written for the new context.  On the other hand, too much abstraction leaves the developer with little ability to adapt to their specific situation.

\section{Technologies}
Due to varied background of those who tend to read senior individualized projects, this section seeks to introduce many of the technologies used and referenced throughout this paper.  Should the reader feel like they have a strong grasp on what the following technologies are, they may skip over those sections without missing content crucial to the rest of the paper.

\subsection{GraphQL}
The main technology motivating this project is GraphQL, a query language developed by Facebook in 2012 \cite{byronKeynoteBriefHistory2019}.  This language, in comparison to querying languages such as SQL (Structured Query Language), is designed to be exposed as a public, web accessible, API.   In this section, I will briefly cover what makes GraphQL different from languages like SQL and how web APIs defined with GraphQL differ from a typical REST API.

\subsubsection{Language Structure}

The unique aspect of GraphQL that when data is requested from the API, the client requests the shape of the response.  What this means is that rather than the server defining the exact data that it will respond to the client with, the client has the freedom to adapt the data requested.

Another unique part of the language is that data accessible by the API is represented as a graph of connected data types.  You could imagine each type of requestable data being a node on a graph where the edges represent relationships between the data.  When combining the ability to dynamically request data from the server and being able to request related data, the power of GraphQL becomes evident.  Since the client requests exactly what it wants and can recieve all the related data in one request, the number of requests to the server is dramatically reduced and no extraneous data is sent to the client.

\begin{figure}[htbp]
    \centering
    \includegraphics[scale=.15]{img/schema-graph.png}
    \caption{INSERT CAPTION HERE}
    \label{fig:schema-graph}
\end{figure}

Figure \ref{fig:schema-graph} is a visual representation of a GraphQL schema with three types, TimestampCollection, TimestampCollectionLabel, and Timestamp.  The relationships on this schema would allow the user to create a request as shown in Figure \ref{fig:basic-query}.  The result of this query would be all TimestampCollections with their associated labels and timestamps.  If not all the fields or connected data is need, however, any part of that query could be removed and the API would respond with less data.

\begin{figure}
    \begin{verbatim}
query {
    timestampCollections {
        identity
        title
        labels {
            title
            timestamps {
                time
            }
        }
    }
}
    \end{verbatim}
    \caption{A basic GraphQL query}
    \label{fig:basic-query}
\end{figure}

\subsubsection{GraphQL vs. SQL}

SQL is debatably one of the most popular querying languages thanks to the popularity of relational databases.  While the aim of this project is to connect GraphQL and SQL queries, it should be noted that the two languages exist for very different reasons.

SQL is a language very tied to the storage and retrival of data.  Queries act on specific tables that have strictly defined and structured data.  The SQL language does not make sense outside of the context of a relational database.  GraphQL on the other hand is completely agnostic to how the data is stored/retrieved.  The data source could be a relational database, a non-relational database, some other API, or could be programatically generated. To fetch and update all of this data, GraphQL servers implement resolvers that have the logic to retrieve each data type/allowed query.  In this sense, GraphQL can be the requesting language from the client side and SQL queries can be used to resolve those GraphQL queries.

\subsubsection{GraphQL vs. REST API}
\subsection{REST API}
In 2000, Roy Fielding wrote a PhD thesis focusing on the software architecture of web applications \cite{fieldingArchitecturalStylesDesign2000}.  In this dissertation, Fielding defined a style of designing web servers called Representational State Transfer (REST).  Since then, REST APIs have been the de facto standard for organizing services that send data over the internet.  Whether it is requesting individual web pages or requesting bits of data to be used in an application, REST design patterns are often used. 

To this day, the basic constraints laid out in Fielding's dissertation are still used as the tenants of well designed RESTful interfaces.  A quick google search of REST principles will bring up hundreds of blog posts and tutorials summarizing these principles. For more information about REST, the reader is encouraged to do this additional research, however, I will include a brief summary below of the basic rest principles.

\subsubsection{REST Principles}

\begin{itemize}
    \item \textbf{Client-Server:} There should be a division between the software that stores and retrieves data and the software that displays the data.
    \item \textbf{Stateless:} All of the data required to respond to the request should be contained in the request. The server should not store the state of its interaction with the client. That is the client's responsibility.
    \item \textbf{Cache:} The results of a server request should be labeled as cacheable or non-cacheable.  Cacheable requests should not be re-requested for a certain time period to improve network efficiency.
    \item \textbf{Uniform Interface:} All REST services should respond in a uniform way, with each resource having a Unique Resource Identifier (URI).  Through these standard interfaces, web browser can go to nearly any web page without specific configuration.
    \item \textbf{Layered System:} The client is only aware of the top most layer of a REST system, but there may be many layers beneath the public facing layer that process the request.
    \item \textbf{Code-On-Demand:} The client can request code that extends the functionality of the application.  This is the basis of how browsers work.  They request code in the form of HTML, CSS, and JavaScript, which run on the machine and also often continue to request more code.
\end{itemize}

Most of these principles are thankfully handled without much interference by the modern day developer. Caching, Layered Systems, and Code-On-Demand, are all key aspects implemented by web browsers and operating systems.  What takes most of the work on the developer's part is the organization of data into a Uniform Interface. In a REST API, if I want to request the TimestampCollection with an id of 1, the URI I use to request that resource might be \Verb"/api/timestampCollections/1".  Often, I would retrieve that information using an HTTP GET request.  An edit to that resource might then come through a PUT or DELETE request.

\subsubsection{REST vs. GraphQL}
For how often GraphQL is talked about as an alternative to REST APIs, it should be noted that many of REST's principles are typically implemented in a GraphQL API.  GraphQL servers respond to requests from client programs, GraphQL requests are often cached by their clients, such as the Apollo GraphQL client, and the web servers that respond to GraphQL requests are stateless and behind layered systems.  The biggest deviations from REST are in the fact that the organization of a GraphQL api is not through a Uniform Interface. In GraphQL all allowed queries and mutations are sent through a single `/graphql` endpoint.  Those requests then have a query name or mutation name attached to them that the GraphQL server interprets and knows how to respond to the client.  This allows for a much more flexible, but necessarily uniform interface; however,  Since all requests are still over HTTP a simple JavaScript client can allow almost any web page to interact with a GraphQL server.
\subsection{Modern JavaScript}

The code for the project was all written in JavaScript or JavaScript variants.  To understand this ecosystem of technologies, a few different aspects must be covered.

\subsubsection{Node.js}
As this project is focused on developing the server-side of web applications, JavaScript and related languages, for many years, would have been seen as an unconventional choice.  The original version of the language was created in 1995 for the Netscape 2.0 browser and was originally prototyped in only 10 days by Brendan Eich \cite{redhatinc.CreatingJavaScript}.  An article published in December of that year claimed that the new language designed for beginner programmers would make it easy for developers to interact with HTML and create web applications \cite{bucholtzNewLanguageAims1995}.  Fourteen years later, however, prompted by the release of Google Chrome, and their JavaScript engine known as V8, Ryan Dahl developed the Node.js runtime \cite{pramodEpisodeInterviewRyan}.  In an interview on the podcast "Mapping the Journey," Dahl saw the asynchronous aspects of the programming language, and the fact that it ran on a single thread, as making the language a perfect canidate for a robust, scaleable, server-side language.  He then adapted the V8 engine, which was designed for a web browser, to run locally on a machine, thus allowing JavaScript code to be executed on a server.
Since then, Node.js and JavaScript have continued to grow in popularity. In this last year's Stack Overflow developer survey JavaScript, was the commonly used programming lanugage and Node.js was the largest non-web/non-browser based frameowrk \cite{stackoverflowStackOverflowDeveloper}.

When discussing Node.js, the most important thing to understand is that it is simply a runtime to run JavaScript on a machine without a full browser.  The framework aspect of it comes through the various standard libraries provided by Node such as their file system APIs.

\subsubsection{TypeScript}
In the modern JavaScript ecosystem, Node.js was developed to adapt JavaScript for a different runtime environment.  A few years later, another technology, TypeScript, was created to adapt JavaScript in order to solve a different issue: static typing.  Starting in 2010, as Microsoft began working on web applications and talekd with their customers, they realized that JavaScript's lack of typings made managing large applications difficult.  One can imagine that as a code base grows, if there is not compliation step, small adjustments to the shape or type of data stored in variables or functions could be hard to catch.  With errors only surfacing at runtime, these applications became tough to maintain.  In an article with IDG when TypeScript, a Microsoft engineer, pointed out these issues and described their soultion, TypeScript, a supserset of JavaScript \cite{idgnewsservicestaffMicrosoftAugmentsJavaScript2012}.

When writing in TypeScript, the syntax is almost nearly identical to JavaScript, but varaible declarations and function definitions also include types.  What makes Typescript somewhat irregular amongst typed languages is that these types are completely removed at compilation time. The advantage to this has been that TypeScript and JavaScript code are completely compatible.  Any TypeScript developer can thus take advantage of existing JavaScript libraries, and JavaScript developers are not required to adopt TypeScript to use these recently developed libraries.

\section{Motivation}

\subsection{Practice Liszt}
My discoveries related to began with a project I called Practice Liszt.  The main goal of the application was to be a rehearsal tool for musicians. As a percussionist, I found it difficult to practice along with video or audio excerpts on applications such as YouTube or Spotify.  As I would try to practice with sometimes 30 or 45 minute long videos, I would have to write down the times of different sections within a piece.  For example, if I wanted to start my practice at the beginning of as second movement of a piece, I would have to write down the time that occurs in the video, scrub to that time, and then start the video with enough time to pick up my instrument/implements and being playing.

My intended solution to the project was to build a web application where I could create timestamps in these lengthy pieces of audio or video and keep a them in an easily accessible list, hence the pun for a title "Practice Liszt." In the Spring of 2018, I did some basic research into the feasibility of this project.  I confirmed that the YouTube API had adequate controls to change the time of YouTube videos, that the interface could be easily created in React.js, a JavaScript UI Framework, and that GraphQL would be an appropriate means of communicating between the front-end (browser) of the application and the back-end (server and database) of the application.

After this initial mock-up, however, I did not work on the project again until the spring of 2019.  At the time I was interviewing for an intern position at the Kalamazoo company Maestro.  For my interview process, I had to implement a project that could show off my skills as a web developer.  Knowing I wanted to tackle this project in the future, I used this as an opportunity to create a complete prototype of the application.  By the end of the week, I had a working prototype that used React, GraphQL, and a non-relational database to store the timestamps called MongoDB.

Thankfully, the software team at Maestro was impressed by my project and I was offered an internship to work there. Going into the summer, I had planned on continuing to work on the project, from a feature standpoint, while I was not working at Maestro.  Ultimately, my goal was to have an application that was not only simple for a musician to use, but also could be used for academic research. On the musician side, they would be able to create timestamps in multiple versions of a piece (think two different recordings, or a video and a piece of audio), they would be able to share those timestamps with others, and they would be able to keep their timestamps private if they wish.  For academics, I hoped that they would be able to look at what pieces have been timestamped and by extension, what recordings are associated with which pieces and what spots in the pieces are commonly timestamped.

To achieve this academic side of the project, I realized in July that that meant redesigning the database and data model for the application.  In order to associate timestamps with actual published pieces of music, I decided to integrate my database with the MusicBrainz database, which is an open source database of music metadata, such as published works, composers, dates, and numerous other bits of data.  This database, however, was structured as a PostgreSQL database.  For this reason, my database went under a complete re-write between July and August to handle data retrieval from a relational SQL database instead of a non-relation database.  It was this process, along with my learning at my internship at Maestro that ultimately led me to become interested in improving the interoperability of GraphQL and SQL.

\subsection{Maestro}
Meanwhile, while working on my Practice Liszt project, I started working at Maestro in June 2019.  In my time at Maestro, so far, one of the biggest takeaways I have had is how costly code duplication can be as projects go.  My main project over the summer was refactoring the back-end code for a learning management system called Loop.  The motivation for this project was to move all of the existing code into a framework called Nest.js.  In this process though, we used the refactor as an opportunity to remove a lot of duplicated logic throughout the code base.  Logic around data validation, processing HTTP requests, and permissions/authorization were all redesigned to make their implementation more DRY (Don't Repeat Yourself) and manageable.

At the end of the processes, however, due to the nature of the technology being used, a lot of logic was still duplicated throughout the entire application.  Since the application has been written to have static REST endpoints, for every endpoint, we have to define exactly what data should be returned and how to fetch that from the database. With over a two hundred endpoints, this creates code that is doing the same thing, just in slightly different ways.  Moreover, whenever the front end of the application needs different data a back end developer has to make a change to the code base to manually include or reformat specific fields.

As I was experiencing some of these frustrations while working on my Practice Liszt application, I knew that I wanted to focus on developing a back-end to my application that would minimize these code repetitions.

\section{GraphQL Query Mapping}

\subsection{Existing Solutions}

As I began working on my solution to the problems encountered above, I did come across some existing solutions.  Here I will briefly describe each existing solution and as I explain my solution, I will compare those aspects to the other libraries.

\subsubsection{Hasura}

Hasura is a complete engine for mapping GraphQL queries to a PostgreSQL relational database. The application manges the data schema, manages authorization/authentication, and creates an actual server to respond to the requests. \cite{hasurainc.HasuraGraphQLEngine}.  By installing the open-source software, the developer gets a web dashboard where the schema is entered and all aspects of the server are managed.  In my experience setting up this solution, it was by far the fastest way to map GraphQL queries to SQL.  Thanks to the web GUI, little documentation is needed to actually define a schema.  The downside, however, is that because the application is defined in the browser, the extensibility of the application is limited to the features already implemented in the software.

\subsubsection{JoinMonster}

JoinMonster is a JavaScript library that is similar to my solution. It allows the developer to define a GraphQL Schema and relate it to SQL tables to automatically create SQL queries that join related tables \cite{carlJoinMonster}.  This solution, being a JavaScript library, is far more extensible.  You can integrate the function to generate SQL queries wherever needed in the structure of your GraphQL server.  In doing this, though, JoinMonster has defined its own mechanism for generating SQL queries instead of employing existing solutions such as an Object Relation Mapping (ORM), which creates objects based on database responses, or a query builder, which simplifies writing SQL queries.  As a result, querying the database elsewhere in the application requires either redefined data models, additional libraries, or manually generated queries.
\subsection{Schema/Data Model}

The first step to mapping GraphQL queries to SQL queries is to define a shared schema between the allowed objects that can be requested in GraphQL and the tables in the relational database.  If the GraphQL schema and the database schema are defined in the same way, then each table in database becomes a GraphQL type that a client can request.

The first major design question was how to define these schemas at the same time.  In the JoinMonster (note: need to introduce existing technologies before this), library, 
\subsection{Generating Queries}

Once the schema between the GraphQL queries and the Database Tables has been established, by using a library called Sequelize, the process of converting GraphQL queries to SQL queries is fairly straight forward.

Sequelize is what is known as an Object Relational Mapping library, or ORM.  ORMs are useful because they abstract the tables of the database as objects that can be used in a given programming language.  In this way, each record is no longer a row in a table, but an object of a given type.  Along with this abstraction, ORMs often provide a means of querying the database using functions instead of constructing raw SQL statements.  By calling functions such as \Verb!findOne! or \Verb!create!, the library will generate SQL statements that correspond to those function calls.

Since the definition of the table schema shown in Section \ref{sec:schema} is compatible with Sequelize, this means that in order to generate a SQL statement for a given GraphQL query, all the program must do is determine what related tables to request and ask Sequelize to include them in a query.

\subsubsection{Example Sequelize Query}

\begin{figure}
    \begin{verbatim}
timestampCollectionRepository.findAll({
    include: [{
        model: TimestampCollectionLabel,
        as: 'labels',
        include: [{
            model: Timestamp,
            as: 'timestamps'
        }]
    }]
})
    \end{verbatim}
    \caption{An example Sequelize query}
    \label{fig:sequelize-query}
\end{figure}

Figure \ref{fig:sequelize-query} is example of what a query might look like in Sequelize.  It will query all of the Timestamp Collections from the database and will also join onto two other tables to get all of the labels for the collection, as well as all of the timestamps associated with each label.  This query which is passed to Sequelize, the ORM that generates SQL queries, would retrieve all of the data required to resolve the GraphQL in Figure \ref{fig:basic-query}, which requests a Timestamp Collection and all of its related Timestamps.  When this code is executed, Sequelize will generate a query to fetch this data and will return the data back in the form of nested objects.  This is also in the format that GraphQL expects, allowing the data to be easily sent back to the client.

\subsubsection{Constructing the ``include'' object}
In order to dynamically fetch the requested data, the \verb!include! object seen in Figure \ref{fig:sequelize-query} must be generated for each request.  By parsing the incoming GraphQL query, one can determine what connected types are requested.  In Figure \ref{fig:basic-query}, the requested types are \verb!TimestampCollection! $\rightarrow$ \verb!TimestampLabel! $\rightarrow$ \Verb!Timestamp!.  Since we also defined Sequelize database models for each of those types, by retrieving the equivalent Sequelize model for each requested GraphQL type, an \verb!include! object can be dynamically created. Algorithm \ref{alg:construct-includes} defines the pseudo code procedure that could generate these \verb!include! objects.  Essentially, when given a GraphQL query, this function will return an equivalent Sequelize query that wil retrieve all of the data required by the query. For the query in Figure \ref{fig:basic-query}, the result of this function would be the object on the \verb!include! key in Figure \ref{fig:sequelize-query}.

\begin{algorithm}[H]
    \begin{algorithmic}
        \Procedure{ConstructIncludes}{$sequelizeModel$, $graphQLSelectionSet$}
            \State $includedModels \leftarrow []$
            \For{$selection \in graphQLSelectionSet$}
                \State $inclueObject$ = \{\}
                \If{$selection$ is type (not a field)}
                    \State $selectedModel \leftarrow$ equivalent Sequelize model for GraphQL Type
                    \State $includeObject.model \leftarrow selectedModel$
                    \State $includeObject.includes \leftarrow$ 
                    \State \textsc{  ConstructIncludes}($selectedModel$, $selection$)
                    \State add $includeObject$ to $includedModels$ array
                \EndIf
            \EndFor
            \State \textbf{return} $includedModels$
        \EndProcedure
    \end{algorithmic}
    \caption{Construct includes object for Sequelize Query}
    \label{alg:construct-includes}
\end{algorithm}

\section{Framework}
With the ability to map queries from GraphQL to SQL, the process of fetching data is reusable throughout the whole application; however, the developer would still have to set up a function known as a resolver to reach out to the database based on the results of the \verb!ConstructIncludes! function.  This would again create duplicated code that this project has sought to remove.  To simplify this process, this project proposes the use of a \verb!BaseEntity!, \verb!BaseService!, and \verb!BaseResolver! that, when used together, automate the process of creating queries and mutations for a given data type, fetching the data from the database, and returning the data to the client.  These classes are created using tools provided by Nest.js (see Section \ref{sec:nest-js}) and the library it uses to create GraphQL severs, Type-GraphQL.  By taking advantage of these libraries, this framework could be integrated into any server built using Nest.js.  Moreover, these classes are extensible to allow custom functionality and extension by the developer.

\subsection{Design Considerations}
The framework is heavily influenced by two design patterns: dependency injection, the pattern coined by Martin Fowler and popularized by the Java Spring framework \cite{fowlerInversionControlContainers2004}, the Bridge pattern outlined in the book \textit{Design Patterns: Elements of Reusable Object Oriented Software} by Gamma, et. al \cite{gammaDesignPatternsElements1995}.  These two patterns when combined allow independent implementations of creating, reading, updating, and deleting data to be implemented for each data type, while a standard interface allows for basic code to be reused across each data type.

\subsubsection{Dependency Injection}
Following the Dependency Injection Pattern is a way of providing dependencies to an instance of the class, rather than them being defined in the class itself \cite{fowlerInversionControlContainers2004}. What this allows is for behavior to change of specific actions implemented by the dependency, while the parent class does not have to change its implementation at all.  In my framework, the clearest use of dependency injection occurs between the \verb!BaseResolver! and the \verb!BaseService!.  A resolver, in a GraphQL Server, is what defines the functions that are available for a client to call.  For example, it may define a query or mutation to create an object, read an object, update an object, or delete an object (these operations are commonly abbreviated as CRUD).  The implementation for each of these actions on the database, however, may be different for each object type.  These implementations live in the subclasses of the \verb!BaseService!.  For a specific object type, a service class will be defined that contains the logic to perform all CRUD operations and an instance of this class is then injected into the resolver for that type.  Thus the resolver doesn't depend on a specific implementation for each type of object; it receives an instance of the service that knows how to modify the object in the database and simply forwards requests todo so to the service instance.

\subsubsection{Bridge Pattern}
Another perspective on the relationship between the \verb!BaseResolver! and \verb!BaseService! can be seen through the lense of the Bridge pattern.  This pattern from Gamma's \textit{Design Patterns}, similar to Dependency Injection aims to separate the specific implementation of abstraction from the contract it provides to those that call it. In this framework, the \verb!BaseResolver! requires that the \verb!BaseService! provides a certain functionality to modify the database.  For each object type, many of the CRUD operations always exist.  The logic to expose these functions and interact with the database live in the \verb!BaseResolver! and \verb!BaseService! classes.  For each object type, however, there must be a concrete resolver that depends on a concrete service. For example, to modify a \verb!TimestampLabel!, there must be a \verb!TimestampLabelResolver! and a \verb!TimestampLabelService!.  By having the two abstract \verb!BaseResolver! and \verb!BaseService! classes, a certain contract is established that for all subtypes of the two classes. The functionality will always exist to create, read, update, and delete data.  In this sense, the base classes are a bridge between concrete implementations. On the other hand, behavior can still be extended in the concrete implementations without duplicating logic provided by the base classes.

\subsection{BaseEntity}

To avoid duplicating logic around retrieving data out the database, it is helpful that all database entities share a common structure.  When I began working at Maestro, I was introduced to this pattern through the software team's use of a parent class called \verb!BaseEntity!.  This data model includes fields such as \verb!id!, \verb!identity! (a Universally Unique Identifier or UUID, instead of a integer id), and \verb!createdAt!, \verb!updatedAt!, and \verb!deletedAt! timestamps.  Once this class has been established all other data models in the application should extend this \verb!BaseEntity!.  This way, for any type of data in the database, I know how to retrieve the records since they all have an id, and identity. I also know that when I create those records, consistent metadata will be kept regarding their creation, modification, and deletion. When creating a service that will handle these operation, this baseline knowledge about each is crucial.

By taking a subclassing approach to the data model, duplicating logic around these duplicated fields can also be avoided.  Looking back to the code example in Appendix \ref{ex:tl-entity}, the code does not specify that a \verb!TimestampLabel! has an \verb!id!, \verb!identity!, etc. Those basic fields are inherited from \verb!CreatedByEntity! which extends \verb!BaseEntity!.  This leads to yet another benefit of subclassing: there can be multiple levels of inheritance. For example, for many tables in the database, the user that creates the record should be tracked as well.  Having a \verb!CreatedByEntity! that extends \verb!BaseEntity! means that the \verb!created_by_id!, \verb!updated_by_id!, and \verb!deleted_by_id! can be added to this one model and never duplicated again throughout the code base.  Other models that are created by a user simply extend the \verb!CreatedByEntity!, thus receiving all \verb!BaseEntity! and \verb!CreatedByEntity! fields.

\subsection{Base Service}
With a standard structure for the data model of each type of data being stored, we can also create a standard set of functions related to creating, reading, updating, or deleting these records.  Once the data model has been established, Sequelize, the library used to define the data model, provides a set of functions that directly interact with the database such as \verb!findOne!, \verb!findAll!, \verb!create!, etc.  The \verb!BaseService! wraps these functions to provide additional functionality, as well as enforce standard interface with which to interact with the data.

\subsubsection{Merge Options}
When making queries to the database with Sequelize, the client is allowed to pass in options to modify the behavior of the query.  For example the options shown in Figure \ref{fig:merge-options} are all examples that either include data of related tables in the request (through an \verb!include! option) or limit the data that is returned (through \verb!where! options).  Throughout the \verb!BaseService! class, it would be useful to is together two different sets of options.  For example, the application may allow the user to search for a timestamp label by its name; however, the user should only have access to labels that they created.  To achieve this, on each call to the database, the application can merge in the option \verb!{ where: {createdById: user.id}}! so all requests have that constraint.  Because options are a mix of objects, such as \verb!where! options, and arrays, such as the \verb!include!, the logic to merge these options becomes somewhat complicated.  The solution I came to, however, is to do a deep merge on object values and to iterate through \verb!include! options and either add a distinct model to the array, or deep merge matching models.  A deep merge is where the keys and values at each level of an object are combined. On each level, if both objects have the same key, their values are merged as well.  On leafs of the tree, meaning individual values instead of nested objects, one object's value is chosen to have the overriding value, and its value is used in the final result.

For certain parts of the query options, however, this default behavior doesn't work.  For \verb!include! options, you can't simply combine the \verb!include! arrays from the merging options.  This would potentially result in related tables being added multiple times.  Instead you have to union the two sets together and merge common tables. For example, if one \verb!include! joins the \verb!TimestampCollection! table and another include joins the the \verb!User! table and the \verb!TimestampCollection! table, the result should join the \verb!User! table and the \verb!TimestampCollection! table where both options for the \verb!TimestampCollection! table have also been merged.

On \verb!where! options, the two \verb!where! clauses are simply wrapped in an \verb!AND! operator.  This allows both constraints to be applied, but no merge has to be computed.

Figure \ref{fig:merge-options} gives an example of what the outcome of this function would be for two example options.

\begin{figure}
    \begin{Verbatim}[fontsize=\footnotesize]
const initialOptions = {
    include: [{
        model: RelatedModel
    }]
    where: {
        createdById: 10
    }
};

const optionsToBeMerged = {
    include: [{
        model: RelatedModel,
        include: [{
            model: ASecondModel
        }],
        where: {
            createdById: 10
        }
    }],
    where: {
        name: 'A Value'
    }
};

const optionsAfterMerge = {
    include: [{
        model: RelatedModel,
        include: [{
            model: ASecondModel
        }],
        where: {
            createdById: 10
        }
    }],
    where: {
        [Op.and]: 
        [
            {
                createdById: 10
            },
            {
                name: 'A Value'
            }
        ]
    }
}
    \end{Verbatim}
    \caption{Example Sequelize Options}
    \label{fig:merge-options}
\end{figure}

\subsubsection{Base Find Options}
With the ability to merge query options, the \verb!BaseService! provides a mechanism of not only overriding the standard behavior of requests on each method call, but it also provides the option to set base-level constraints on each query.  When extending the \verb!BaseService! class, the subclass can override a function called \verb!getBaseFindOptions!.  This function is called for every fetch to request data from the database and the options it returns are merged into the options provided by the caller.  A common use case for this would be to use this function to ensure that the user can only access records they created.  Looking at the example service in Appendix \ref{ex:tl-service}, we can see how crucial this ability to limit access is.  For Timestamp Labels, they should only be viewable by the user that created that label, or by any user if its collection is marked as public.  By defining a \verb!getBaseFindOptions! method on an implementing service, that logic is applied to all access operations on this object type.  

This is one of the largest benefits of this framework versus Hasura or JoinMonster.  Since functionality is kept in extendable classes, this kind of behavior can be modified for each type defined in the application. Additionally, because the data model defined in the entity is defined for a fully functional ORM, the developer can use that same data model to modify the functionality of their service. No re-definition has to occur.

\subsection{Base Resolver}
The resolver is what exposes all of the functionality of the services to clients through GraphQL queries and mutations. The entities and services have created all of the functionality to interact with the database, but they do not provide any public APIs for clients to take advantage of this functionality.  The resolver is what creates public functions that clients call to interact with data.  These function calls are in the GraphQL language.  The server then processes these requests and forwards them to the resolver, which is in charge of fetching or mutating the requested data.  The \verb!BaseResolver! is a class that automatically provides CRUD operations for a given data type when supplied a service class that extends the \verb!BaseService!. The idea for the \verb!BaseResolver! comes from the TypeGraphQL documentation \cite{lytekTypegraphql2019}, where they have an example to create a \verb!BaseResolver!.  In my version of this example, I have integrated it with the \verb!BaseService! and also allowed the developer to turn on and off the functions defined by the \verb!BaseResolver!. By default, a resolver extending the \verb!BaseResolver! will receive the following queries from the \verb!BaseResolver! class: \verb!getOne! and \verb!getAll!.  It will also have the following GraphQL mutations: \verb!create!, \verb!update! and \verb!delete!.  When implementing a resolver that extends a \verb!BaseResolver!, all the developer must provide to the resolver is a service that extends the \verb!BaseService! that will fetch and mutate the data managed by the resolver.

\subsubsection{Connecting the Dots}

The resolver ultimately connects all aspects of the application together.  When a request is received, Nest.js sends the request to the associated resolving function for that request, which will either be defined by the \verb!BaseResolver! for simple CRUD functionality, or will be defined in an implementing resolver for custom functionality.  For all of the requests built into the \verb!BaseResolver!, the function then retrieves the requested data by calling the \verb!ConstructIncludes! function and passing that query into the service which will actually fetch the data from the database.  That resolver then returns that data to the GraphQL server which forwards the result back to the client.

\subsubsection{Custom Functionality}

When the developer extends the \verb!BaseResolver! to create a concrete resolver for a specific type, it is here that they have specific control over what queries and mutations are available on the server.  In this implementing class, the developer can turn on and off creating, updating, and deleting of the object, they can set guards to limit access to certain functions, they can they can implement custom queries, and they can override any functionality defined in the \verb!BaseResolver!.

\subsection{Timestamp Label Example}
Appendix A is an example of the code required to enable CRUD operations for the Timestamp Labels by the GraphQL server for the Practice Liszt app.  In the app, a Timestamp Label defines the name of a bookmark in a video or piece of music.  The main data that has to be collected for a Timestamp Label are the associated collection and the name of the label.  Below I will point out some of the customizations that the framework has enabled and where those features are implemented.

\subsubsection{Data Model/Custom Fields}
In the \verb!TimestampLabelEntity! (Appendix \ref{ex:tl-entity}), the required fields for the Timestamp Label are defined.  Those fields are \verb!timestampCollectionId!, which is used to tie the label to a Timestamp Collection, \verb!title!, and \verb!position!, which allows the user to arrange timestamp labels in the order they occur.  Additionally, a relationship is defined between the Timestamp Label and its collection.  The \verb!title! and \verb!position! fields are marked with \verb!@Field()! decorator, as well, which allows the client to request those specific properties on a GraphQL query.  In this way, the developer has complete control over what data is actually public to the user.  In this example, the timestampCollectionId will never be accessible to the client.

\subsubsection{Creation}
Since the Timestamp Label has to be associated with a Timestamp Collection, the developer has overridden the create function \verb!create! function for the Timestamp Label Service (see Appendix \ref{ex:tl-service}). This is an example of modifying the behavior of creating an object type in the database.  Every time a Timestamp Label is created, the server finds its associated Timestamp Collection, ensures the user has permission to add labels to that Timestamp Collection, and increments the number of labels on that Timestamp Collection.

\subsubsection{Access Permissions}
The Timestamp Label service also limits the Timestamp Collections that a user is allowed to find (see Appendix \ref{ex:tl-service}).  In the \verb!getBaseFindOptions! function, which overrides the \verb!BaseService!, the developer defines a base query that will limit the available Timestamp Labels to either labels on Timestamp Collections that the user created, or Timestamp Collections that are explicitly marked as public.  This restriction is applied to all queries on the database for this object type.

\subsubsection{Modification Permissions}
On a more specific level, in the Timestamp Label resolver, the developer limits the options used to find a Timestamp Label to be updated or deleted (see Appendix \ref{ex:tl-resolver}).  In this situation, the user will only be able to modify Timestamp Labels that they created.  This is more restrictive than the base find options in the service which allow the user to find Timestamp Labels that are in public collections, whereas for modifications, they should not have access to those public collections that they did not create.

\section{Query Mapping Limitations}
The main limitation on this project is that query mapping as it stands is incomplete. As implemented, the algorithm assumes that there is only one GraphQL request being made at a time.  In the GraphQL specification, and in Nest.js's implementation, multiple GraphQL requests can be sent at once and the server will respond once with the result of both queries.  If multiple queries are sent, the \verb!ConstructsInclude! function would return the includes options for the first query for all requested queries.

To adjust the \verb!ConstructIncludes!, instead of just inspecting the top level query, the function will instead analyze each query separately.  The major difficulty is that the caller will need to tell the \verb!ConstructIncludes! function which query should be analyzed. Typically, this would happen in the resolver class when it calls the \verb!ConstructIncludes! function. The first course of action would be to attempt to retrieve this information from Nest.js.  Somehow Nest.js knows which resolver functions to call based on the GraphQL queries it receives, so it must have some metadata stored about which query is being request by the client.  Once it is determined how to retrieve this information from Nest.js, the adjustment to the \verb!ConstructsIncludes! can be completed.

Additionally, the query mapping does not currently support union types.  These would occur if the server could return multiple different object types on a single connection.  The example in the GraphQL documentation for union types is a search that could return three different types.  The GraphQL request for that function would then specify what fields they want for each those types.  The problem with trying to support union types is that they do not fit well with relational databases.  In a relational database, it is known what the type of each related data is.  At the core database level, it is not possible to store a union type.  There is a way to simulate this flexibility which is known as polymorphic table design.  In this situation, you store a reference to the associated data and store another field denoting what type of data is in the referenced table.  This breaks the ability to have those references as foreign keys in the database, which is the main building block of relational databases. These tables can be useful, but ultimately present significant challenges to traditional relational database design.  For this reason, mapping GraphQL union types to a relational database becomes a complicated task.  As long as this project continues to focus on mapping GraphQL queries to SQL queries, this problem will likely not be a major focus.

Overall, the project works as I hoped it would, but it requires the user to limit their use of GraphQL.  Additionally, developers must be aware of these limitations.  If this project were open-sourced today and other developers began using the framework, it would likely be a frustrating experience discovering what features they could use and what they could not.  To address these issues, additional research into the GraphQL Abstract Syntax Tree and how to parse computer languages in general, will need to be conducted and documented.

\section{Future Work}

The majority of future work is centered around the customizing the resolver.  When developing an application, it is common to need the ability to dynamically filter and change the data set being returned by a query.  For example you might want to change the order that data is returned, only want to view a page's worth of data, or you may only want to query objects where a certain condition is true or false.  Each one of these actions would be prompted by different parameters on the query.  In this section, I will propose how those features might look for the developer once they are fully implemented.

\subsection{Pagination}

A common feature of large applications is viewing paginated subsets of data.  For example, in the Practice Liszt app, there might be thousands of public timestamp collections, but the user can only view ten at a time.  The GraphQL server, then needs a way to deliver these thousands of records in pages to the server.  I believe this feature will be possible to integrate into my framework, but it will take some work.  At the resolver level, the number of results and the page to retrieve will have to become arguments for each paginated query. These options will then be merged into the service's options to only query the database for that subset of data.  Additionally, a default sort order will have to be chosen before paginating the data.  If no default sort order is determined, the database doesn't know what order the "pages" are in.  If the order is left random, "Page 2" will be different every time.  By having a default sort order, as the client requests subsequent pages, only new data will be returned on each request.

\subsection{Sorting}

In addition to a default sort order, it will also be necessary to allow the client to modify the sort order of the results.  In a web application, it is common to change sorting from alphabetical by name, to reverse alphabetical, or sometimes a different property such as creation time.  The goal would be on the top level of the query, the client would be able to provide an argument that specifies from a list of pre-defined sortable properties which one will be used.

Where these sortable properties could be defined could occur in two places.  Option one would be in the resolver.  This could become another option of the base resolver where the developer simply provides a list of sortable properties.  The resolver then provides those as possible arguments to the client.  What I think would be a more elegant, but more complicated solution, would be to have a decorator on the entity that marks those database fields sortable.  The resolver will then look at the entity to see which fields it has as sortable and will then use those for the list of acceptable sorting properties.

\subsection{Filtering}

The most complicated bit of key functionality left to implement is filtering data based on arguments passed to the query.  An example of this would be searching for a certain Timestamp Label.  On the query, you might give it a string to match and only Timestamp Labels that match that key will be included in the data set.

The difficulty in this use case is determining where to define the filtering behavior and at what step in the process to actually inject the filter into the query.  Ideally the definition of the filter will be on the resolver. At this level, the developer will declare what arguments the client has available to them and will also define what behavior the filter will have.  Applying the filter to the query will have to occur in the \verb!ConstructIncludes! function, however.  Since multiple filters could be applied at multiple levels of the query, it is not enough to just process the top level arguments.  The construct includes function will need to at each level parse the parameters and determine what filters should be applied.  It will then inspect the metadata applied to the resolver to retrieve the defined behavior.  That behavior will be executed and then inserted into the query.  The parsing problem in this situation is similar to sorting, but since the behavior is far more dynamic than sorting, the implementation within that function will be much more complicated.

\section{Conclusion}

Designing a new framework for developing web application back-ends is no small task.  Open source communities of thousands of people support some of the largest frameworks that have taken years to develop. Throughout the paper I have used this idea of a framework to describe the three key reusable entities I have developed, \verb!BaseEntity!, \verb!BaseService!, and \verb!BaseResolver!.  What these classes should provide to the developer is a clear and structured way to implement their GraphQL servers.  Additionally, by using this base structure, some basic functionality surrounding reading and writing the data will also be available by default.

In contrast to many frameworks, however, this project relies heavily on other large existing frameworks, notably Nest.js, Type-GraphQL, Sequelize, and Sequelize Typescript.  Rather than trying to re-write these projects to work exactly as desired, this project augments existing frameworks in specific ways to work together and speed up development time while reducing repeated code.

Looking back to the summer when this project developed, it was my frustration with the amount of repeated code that I found across my personal and work projects that led me to devote my time to investigating automated data fetching.  By providing a set of base classes to use throughout an application, this framework has the potential to reduce duplicated code related to creating, reading, updating, and deleting data from a database.  It is also flexible enough to leave room for some custom implementation.  This is necessary since each app will be developed in its own context, but it also leaves room for misuse.  A developer could ignore the patterns provided by the three base classes and continue to duplicate logic throughout the application.  As is always the case with programming, clean and maintainable code will only come through a combination of design and discipline in implementation.

As we continue to refine our development process at Maestro, creating more re-usable patterns will be our goal moving forward.  Even across projects, we want to move to have one clear way to implement as many of the commonly shared features as possible.  This way, those apps that do not require custom functionality will no longer have to re-implement the same features repeatedly.  This project simply focuses on data mutation and data retrieval using SQL and GraphQL, but additional features could be implemented in a reusable way.  A code library with basic implementations for features including authentication, data retrieval, and error handling would enable us to develop applications for clients at a speed and consistency we have not been able to achieve previously.
\newpage
\setcounter{page}{1}
\newpage
\appendix 
\section{Code Example: Timestamp Label} 

\subsection{timestamp-label.entity.ts}\label{ex:tl-entity}

\lstinputlisting[
    basicstyle=\footnotesize,
    numbers=left
]{code/timestamp-label/timestamp-label.entity.ts}

\subsection{timestamp-label.service.ts}\label{ex:tl-service}

\lstinputlisting[
    basicstyle=\footnotesize,
    numbers=left
]{code/timestamp-label/timestamp-label.service.ts}

\subsection{timestamp-label.resolver.ts}\label{ex:tl-resolver}

\lstinputlisting[
    basicstyle=\footnotesize,
    numbers=left
]{code/timestamp-label/timestamp-label.resolver.ts}


\newpage
\printbibliography
\addcontentsline{toc}{section}{References}
%\bibliographystyle{plain}
%\bibliography{references}
\end{document}

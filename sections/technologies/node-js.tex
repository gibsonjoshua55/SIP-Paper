\subsection{Modern JavaScript}

The code for the project was all written in JavaScript or JavaScript variants.  To understand this ecosystem of technologies, a few different aspects must be covered.

\subsubsection{Node.js}
As this project is focused on developing the server-side of web applications, JavaScript and related languages, for many years, would have been seen as an unconventional choice.  The original version of the language was created in 1995 for the Netscape 2.0 browser and was originally prototyped in only 10 days by Brendan Eich \cite{redhatinc.CreatingJavaScript}.  An article published in December of that year claimed that the new language designed for beginner programmers would make it easy for developers to interact with HTML and create web applications \cite{bucholtzNewLanguageAims1995}.  Fourteen years later, however, prompted by the release of Google Chrome, and their JavaScript engine known as V8, Ryan Dahl developed the Node.js runtime \cite{pramodEpisodeInterviewRyan}.  In an interview on the podcast "Mapping the Journey," Dahl saw the asynchronous aspects of the programming language, and the fact that it ran on a single thread, as making the language a perfect canidate for a robust, scaleable, server-side language.  He then adapted the V8 engine, which was designed for a web browser, to run locally on a machine, thus allowing JavaScript code to be executed on a server.
Since then, Node.js and JavaScript have continued to grow in popularity. In this last year's Stack Overflow developer survey JavaScript, was the commonly used programming lanugage and Node.js was the largest non-web/non-browser based frameowrk \cite{stackoverflowStackOverflowDeveloper}.

When discussing Node.js, the most important thing to understand is that it is simply a runtime to run JavaScript on a machine without a full browser.  The framework aspect of it comes through the various standard libraries provided by Node such as their file system APIs.

\subsubsection{TypeScript}
In the modern JavaScript ecosystem, Node.js was developed to adapt JavaScript for a different runtime environment.  A few years later, another technology, TypeScript, was created to adapt JavaScript in order to solve a different issue: static typing.  Starting in 2010, as Microsoft began working on web applications and talekd with their customers, they realized that JavaScript's lack of typings made managing large applications difficult.  One can imagine that as a code base grows, if there is not compliation step, small adjustments to the shape or type of data stored in variables or functions could be hard to catch.  With errors only surfacing at runtime, these applications became tough to maintain.  In an article with IDG when TypeScript, a Microsoft engineer, pointed out these issues and described their soultion, TypeScript, a supserset of JavaScript \cite{idgnewsservicestaffMicrosoftAugmentsJavaScript2012}.

When writing in TypeScript, the syntax is almost nearly identical to JavaScript, but varaible declarations and function definitions also include types.  What makes Typescript somewhat irregular amongst typed languages is that these types are completely removed at compilation time. The advantage to this has been that TypeScript and JavaScript code are completely compatible.  Any TypeScript developer can thus take advantage of existing JavaScript libraries, and JavaScript developers are not required to adopt TypeScript to use these recently developed libraries.
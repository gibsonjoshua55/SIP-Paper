\subsection{Modern JavaScript}

The original version of JavaScript was created in 1995 for the Netscape 2.0 browser and was prototyped in only 10 days by Brendan Eich \cite{redhatinc.CreatingJavaScript}.  An article published in December of that year claimed that the new language designed for beginner programmers would make it easy for developers to interact with HTML and create web applications \cite{bucholtzNewLanguageAims1995}.  Since then, the language evolved greatly from that initial design.  The syntax has expanded, new runtimes have become standard, and whole variants of the language have been released and skyrocketed in popularity.  In this section, I aim to cover a few of the general developments in JavaScript as well as specific technologies that are being utilized in my project.

\subsubsection{Node.js}
When designing applications that run on servers and machines, instead of in browsers, JavaScript for many years would have been seen as an unconventional choice.  Fourteen years later after JavaScript's release, however, the release of Node.js changed this paradigm as running JavaScript outside of browsers became viable. When the Google Chrome project released their significantly improved V8 runtime for JavaScript, the engine was adapted into a server side execution environment for JavaScript \cite{pramodEpisodeInterviewRyan}.  In an interview on the podcast "Mapping the Journey," Ryan Dahl, the creator of Node.js, claimed that he saw the asynchronous aspects of JavaScript, and the fact that it ran on a single thread, as making the language a perfect canidate for a robust, scaleable, server-side language.  Just by adapting the browser's JavaScript engine to run locally on a machine, he unleashed JavaScript to the server environment.

Since then, Node.js and JavaScript have continued to grow in popularity. In last year's Stack Overflow developer survey, JavaScript was the most commonly used programming lanugage and Node.js was the largest non-web/non-browser  frameowrk \cite{stackoverflowStackOverflowDeveloper}.

When discussing Node.js, the most important thing to understand is that it is simply a runtime to execute JavaScript on a machine without a using browser.  In practice, it has enabled JavaScript to become a first class scripting language with full access to file systems, compiled libraries, and a dedicated developer community.

\subsubsection{TypeScript}
While Node.js was adapting JavaScript for the server environment, a few years later, another technology, TypeScript, was created to adapt JavaScript to solve a different issue: static typing.  In it's pure form, JavaScript is not a typed language.  All variables are dynamic in type, which allows for great flexibility.  As Microsoft began working on web applications and talking with their customers, however, they realized that JavaScript's lack of typings made managing large applications difficult.  One can imagine that as a code base grows, if there is no compliation step, small adjustments to the shape or type of data stored in variables or functions could be hard to catch.  With errors only surfacing at runtime, these applications became tough to maintain.  Microsoft's solution to this, was to create TypeScript, a statically typed JavaScript variant \cite{idgnewsservicestaffMicrosoftAugmentsJavaScript2012}.

When writing in TypeScript, the syntax is nearly identical to JavaScript except varaible declarations and function definitions also include types.  Before running a TypeScript program, this code is ran through a compiler to ensure that all code is correctly using these typed objects. What makes TypeScript somewhat irregular amongst typed languages is that these types are completely removed at compilation time. The advantage to this has been that TypeScript and JavaScript code are completely compatible.  Any TypeScript developer can thus take advantage of existing JavaScript libraries, and JavaScript developers are not required to adopt TypeScript to use these recently developed libraries.  As the developer writes their code, however, these types still exist, so IDEs and text editors can take advantage of the existing types to perform auto completion and check types inline.

Along with typings, TypeScript has also made other syntactic improvements over pure JavaScript.  The feature that this project utilzes the most has been the decorator syntax.  Introduced in version 1.5 of the language, TypeScript included this sytnax while the feature was still in the proposal stage for future JavaScript support \cite{turnerAnnouncingTypeScript2015}.

In programming languages, reflection is the ability for a program to analyze and modify itself during run time \cite{malenfantTutorialBehavioralReflection1996}.  In TypeScript, one of the common forms of reflection is using decorators to attach metadata to classes, properties, methods, and parameters.  By using a package called \Verb{'reflect-metadata'}, the program can reflect on itself to process the attached metadata and modify its behavior.  To illustrate this idea, I will introduce the server framework used in this project, Nest.js, which heavily relies on the use of decorators.

\subsection{Nest.js} \label{sec:nest-js}
Nest.js is a framework for creating server-side applications in JavaScript or TypeScript.  Along with the Angular, thefront-end framework that has heavily influenced Nest.js's design, the framework is object oriented and uses metadata and decorators to declare the structure of the web server.

For example, imagine a simple web server that just responds to one HTTP GET request with a success message.  To define this endpoint, in Nest.js, the program must simply define a Controller, a class that responds to a number of HTTP requests, and specific method to handle that request.

\begin{figure}
    \begin{verbatim}
import { Controller, Get } from '@nestjs/common';

@Controller('/api')
export class AppController {

    @Get()
    async respondToGetRequest() {
       return {success: true};
    }
}
    \end{verbatim}
    \caption{A simple Nest.js Controller}
    \label{fig:nest-controller}
\end{figure}

In Figure-\ref{fig:nest-controller}, we see two examples of decorators: \Verb{@Controller} and \Verb{@Get}.  The former, is what tells Nest.js that this class will have functions that respond to HTTP requests on the route \Verb{'/api'}.  The latter marks the function as the code that will return a response for a GET request on the route \Verb{'/api'}.  When these decorators are excuted at runtime, they store metadata flagging the class as a controller and the funciton as an HTTP request method.  The Nest.js framework then reflects on the established metadata and routes the recieved HTTP requests to the appropriate functions.

As the framework has developed, numerous other features have been implemented and abstracted using decorators, such as request validation, authorization, and GraphQL definitions and resolving.
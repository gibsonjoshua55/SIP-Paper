\subsection{Generating Queries}

Once the schema between the GraphQL queries and the Database Tables has been established, by using a library called Sequelize, the process of converting GraphQL queries to SQL queries is fairly straight forward.

Sequelize is what is known as an Object Relational Mapping library, or ORM.  ORMs are useful because they abstract the tables of the database as objects that can be used in a given programming language.  In this way, each record is no longer a row in a table, but an object of a given type.  Along with this abstraction, ORMs often provide a means of querying the database using functions instead of constructing raw SQL statements.  By calling functions such as \Verb!findOne! or \Verb!create!, the library will generate SQL statements that correspond to those function calls.

Since the definition of the table schema shown in Section \ref{sec:schema} is compatible with Sequelize, this means that in order to generate a SQL statement for a given GraphQL query, all the program must do is determine what related tables to request and ask Sequelize to include them in a query.

\subsubsection{Example Sequelize Query}

\begin{figure}
    \begin{verbatim}
timestampCollectionRepository.findAll({
    include: [{
        model: TimestampCollectionLabel,
        as: 'labels',
        include: [{
            model: Timestamp,
            as: 'timestamps'
        }]
    }]
})
    \end{verbatim}
    \caption{An example Sequelize query}
    \label{fig:sequelize-query}
\end{figure}

Figure \ref{fig:sequelize-query} is example of what a query might look like in Sequelize.  It will query all of the Timestamp Collections from the database and will also join onto two other tables to get all of the labels for the collection, as well as all of the timestamps associated with each label.  This query which is passed to Sequelize, the ORM that generates SQL queries, would retrieve all of the data required to resolve the GraphQL in Figure \ref{fig:basic-query}, which requests a Timestamp Collection and all of its related Timestamps.  When this code is executed, Sequelize will generate a query to fetch this data and will return the data back in the form of nested objects.  This is also in the format that GraphQL expects, allowing the data to be easily sent back to the client.

\subsubsection{Constructing the ``include'' object}
In order to dynamically fetch the requested data, the \verb!include! object seen in Figure \ref{fig:sequelize-query} must be generated for each request.  By parsing the incoming GraphQL query, one can determine what connected types are requested.  In Figure \ref{fig:basic-query}, the requested types are \verb!TimestampCollection! $\rightarrow$ \verb!TimestampLabel! $\rightarrow$ \Verb!Timestamp!.  Since we also defined Sequelize database models for each of those types, by retrieving the equivalent Sequelize model for each requested GraphQL type, an \verb!include! object can be dynamically created. Algorithm \ref{alg:construct-includes} defines the pseudo code procedure that could generate these \verb!include! objects.  Essentially, when given a GraphQL query, this function will return an equivalent Sequelize query that wil retrieve all of the data required by the query. For the query in Figure \ref{fig:basic-query}, the result of this function would be the object on the \verb!include! key in Figure \ref{fig:sequelize-query}.

\begin{algorithm}[H]
    \begin{algorithmic}
        \Procedure{ConstructIncludes}{$sequelizeModel$, $graphQLSelectionSet$}
            \State $includedModels \leftarrow []$
            \For{$selection \in graphQLSelectionSet$}
                \State $inclueObject$ = \{\}
                \If{$selection$ is type (not a field)}
                    \State $selectedModel \leftarrow$ equivalent Sequelize model for GraphQL Type
                    \State $includeObject.model \leftarrow selectedModel$
                    \State $includeObject.includes \leftarrow$ 
                    \State \textsc{  ConstructIncludes}($selectedModel$, $selection$)
                    \State add $includeObject$ to $includedModels$ array
                \EndIf
            \EndFor
            \State \textbf{return} $includedModels$
        \EndProcedure
    \end{algorithmic}
    \caption{Construct includes object for Sequelize Query}
    \label{alg:construct-includes}
\end{algorithm}
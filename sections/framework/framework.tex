\section{Framework}
With the ability to map queries from GraphQL to SQL, the process of fetching data is reusable throughout the whole application; however, the developer would still have to set up a function known as a resolver to reach out to the database based on the results of the \verb!ConstructIncludes! function.  This would again create duplicated code that this project has sought to remove.  To simplify this process, this project proposes the use of a \verb!BaseEntity!, \verb!BaseService!, and \verb!BaseResolver! that, when used together, automate the process of creating queries and mutations for a given data type, fetching the data from the database, and returning the data to the client.  These classes are created using tools provided by Nest.js (see Section \ref{sec:nest-js}) and the library it uses to create GraphQL severs, Type-GraphQL.  By taking advantage of these libraries, this framework could be integrated into any server built using Nest.js.  Moreover, these classes are extensible to allow custom functionality and extension by the developer.

\subsection{Design Considerations}
The framework is heavily influenced by two design patterns: dependency injection, the pattern coined by Martin Fowler and popularized by the Java Spring framework \cite{fowlerInversionControlContainers2004}, the Bridge pattern outlined in the book \textit{Design Patterns: Elements of Reusable Object Oriented Software} by Gamma, et. al \cite{gammaDesignPatternsElements1995}.  These two patterns when combined allow independent implementations of creating, reading, updating, and deleting data to be implemented for each data type, while a standard interface allows for basic code to be reused across each data type.

\subsubsection{Dependency Injection}
Following the Dependency Injection Pattern is a way of providing dependencies to an instance of the class, rather than them being defined in the class itself \cite{fowlerInversionControlContainers2004}. What this allows is for behavior to change of specific actions implemented by the dependency, while the parent class does not have to change its implementation at all.  In my framework, the clearest use of dependency injection occurs between the \verb!BaseResolver! and the \verb!BaseService!.  A resolver, in a GraphQL Server, is what defines the functions that are available for a client to call.  For example, it may define a query or mutation to create an object, read an object, update an object, or delete an object (these operations are commonly abbreviated as CRUD).  The implementation for each of these actions on the database, however, may be different for each object type.  These implementations live in the subclasses of the \verb!BaseService!.  For a specific object type, a service class will be defined that contains the logic to perform all CRUD operations and an instance of this class is then injected into the resolver for that type.  Thus the resolver doesn't depend on a specific implementation for each type of object; it receives an instance of the service that knows how to modify the object in the database and simply forwards requests todo so to the service instance.

\subsubsection{Bridge Pattern}
Another perspective on the relationship between the \verb!BaseResolver! and \verb!BaseService! can be seen through the lense of the Bridge pattern.  This pattern from Gamma's \textit{Design Patterns}, similar to Dependency Injection aims to separate the specific implementation of abstraction from the contract it provides to those that call it. In this framework, the \verb!BaseResolver! requires that the \verb!BaseService! provides a certain functionality to modify the database.  For each object type, many of the CRUD operations always exist.  The logic to expose these functions and interact with the database live in the \verb!BaseResolver! and \verb!BaseService! classes.  For each object type, however, there must be a concrete resolver that depends on a concrete service. For example, to modify a \verb!TimestampLabel!, there must be a \verb!TimestampLabelResolver! and a \verb!TimestampLabelService!.  By having the two abstract \verb!BaseResolver! and \verb!BaseService! classes, a certain contract is established that for all subtypes of the two classes. The functionality will always exist to create, read, update, and delete data.  In this sense, the base classes are a bridge between concrete implementations. On the other hand, behavior can still be extended in the concrete implementations without duplicating logic provided by the base classes.

\subsection{BaseEntity}

To avoid duplicating logic around retrieving data out the database, it is helpful that all database entities share a common structure.  When I began working at Maestro, I was introduced to this pattern through the software team's use of a parent class called \verb!BaseEntity!.  This data model includes fields such as \verb!id!, \verb!identity! (a Universally Unique Identifier or UUID, instead of a integer id), and \verb!createdAt!, \verb!updatedAt!, and \verb!deletedAt! timestamps.  Once this class has been established all other data models in the application should extend this \verb!BaseEntity!.  This way, for any type of data in the database, I know how to retrieve the records since they all have an id, and identity. I also know that when I create those records, consistent metadata will be kept regarding their creation, modification, and deletion. When creating a service that will handle these operation, this baseline knowledge about each is crucial.

By taking a subclassing approach to the data model, duplicating logic around these duplicated fields can also be avoided.  Looking back to the code example in Appendix \ref{ex:tl-entity}, the code does not specify that a \verb!TimestampLabel! has an \verb!id!, \verb!identity!, etc. Those basic fields are inherited from \verb!CreatedByEntity! which extends \verb!BaseEntity!.  This leads to yet another benefit of subclassing: there can be multiple levels of inheritance. For example, for many tables in the database, the user that creates the record should be tracked as well.  Having a \verb!CreatedByEntity! that extends \verb!BaseEntity! means that the \verb!created_by_id!, \verb!updated_by_id!, and \verb!deleted_by_id! can be added to this one model and never duplicated again throughout the code base.  Other models that are created by a user simply extend the \verb!CreatedByEntity!, thus receiving all \verb!BaseEntity! and \verb!CreatedByEntity! fields.

\subsection{Base Service}
With a standard structure for the data model of each type of data being stored, we can also create a standard set of functions related to creating, reading, updating, or deleting these records.  Once the data model has been established, Sequelize, the library used to define the data model, provides a set of functions that directly interact with the database such as \verb!findOne!, \verb!findAll!, \verb!create!, etc.  The \verb!BaseService! wraps these functions to provide additional functionality, as well as enforce standard interface with which to interact with the data.

\subsubsection{Merge Options}
When making queries to the database with Sequelize, the client is allowed to pass in options to modify the behavior of the query.  For example the options shown in Figure \ref{fig:merge-options} are all examples that either include data of related tables in the request (through an \verb!include! option) or limit the data that is returned (through \verb!where! options).  Throughout the \verb!BaseService! class, it would be useful to is together two different sets of options.  For example, the application may allow the user to search for a timestamp label by its name; however, the user should only have access to labels that they created.  To achieve this, on each call to the database, the application can merge in the option \verb!{ where: {createdById: user.id}}! so all requests have that constraint.  Because options are a mix of objects, such as \verb!where! options, and arrays, such as the \verb!include!, the logic to merge these options becomes somewhat complicated.  The solution I came to, however, is to do a deep merge on object values and to iterate through \verb!include! options and either add a distinct model to the array, or deep merge matching models.  A deep merge is where the keys and values at each level of an object are combined. On each level, if both objects have the same key, their values are merged as well.  On leafs of the tree, meaning individual values instead of nested objects, one object's value is chosen to have the overriding value, and its value is used in the final result.

For certain parts of the query options, however, this default behavior doesn't work.  For \verb!include! options, you can't simply combine the \verb!include! arrays from the merging options.  This would potentially result in related tables being added multiple times.  Instead you have to union the two sets together and merge common tables. For example, if one \verb!include! joins the \verb!TimestampCollection! table and another include joins the the \verb!User! table and the \verb!TimestampCollection! table, the result should join the \verb!User! table and the \verb!TimestampCollection! table where both options for the \verb!TimestampCollection! table have also been merged.

On \verb!where! options, the two \verb!where! clauses are simply wrapped in an \verb!AND! operator.  This allows both constraints to be applied, but no merge has to be computed.

Figure \ref{fig:merge-options} gives an example of what the outcome of this function would be for two example options.

\begin{figure}
    \begin{Verbatim}[fontsize=\footnotesize]
const initialOptions = {
    include: [{
        model: RelatedModel
    }]
    where: {
        createdById: 10
    }
};

const optionsToBeMerged = {
    include: [{
        model: RelatedModel,
        include: [{
            model: ASecondModel
        }],
        where: {
            createdById: 10
        }
    }],
    where: {
        name: 'A Value'
    }
};

const optionsAfterMerge = {
    include: [{
        model: RelatedModel,
        include: [{
            model: ASecondModel
        }],
        where: {
            createdById: 10
        }
    }],
    where: {
        [Op.and]: 
        [
            {
                createdById: 10
            },
            {
                name: 'A Value'
            }
        ]
    }
}
    \end{Verbatim}
    \caption{Example Sequelize Options}
    \label{fig:merge-options}
\end{figure}

\subsubsection{Base Find Options}
With the ability to merge query options, the \verb!BaseService! provides a mechanism of not only overriding the standard behavior of requests on each method call, but it also provides the option to set base-level constraints on each query.  When extending the \verb!BaseService! class, the subclass can override a function called \verb!getBaseFindOptions!.  This function is called for every fetch to request data from the database and the options it returns are merged into the options provided by the caller.  A common use case for this would be to use this function to ensure that the user can only access records they created.  Looking at the example service in Appendix \ref{ex:tl-service}, we can see how crucial this ability to limit access is.  For Timestamp Labels, they should only be viewable by the user that created that label, or by any user if its collection is marked as public.  By defining a \verb!getBaseFindOptions! method on an implementing service, that logic is applied to all access operations on this object type.  

This is one of the largest benefits of this framework versus Hasura or JoinMonster.  Since functionality is kept in extendable classes, this kind of behavior can be modified for each type defined in the application. Additionally, because the data model defined in the entity is defined for a fully functional ORM, the developer can use that same data model to modify the functionality of their service. No re-definition has to occur.

\subsection{Base Resolver}
The resolver is what exposes all of the functionality of the services to clients through GraphQL queries and mutations. The entities and services have created all of the functionality to interact with the database, but they do not provide any public APIs for clients to take advantage of this functionality.  The resolver is what creates public functions that clients call to interact with data.  These function calls are in the GraphQL language.  The server then processes these requests and forwards them to the resolver, which is in charge of fetching or mutating the requested data.  The \verb!BaseResolver! is a class that automatically provides CRUD operations for a given data type when supplied a service class that extends the \verb!BaseService!. The idea for the \verb!BaseResolver! comes from the TypeGraphQL documentation \cite{lytekTypegraphql2019}, where they have an example to create a \verb!BaseResolver!.  In my version of this example, I have integrated it with the \verb!BaseService! and also allowed the developer to turn on and off the functions defined by the \verb!BaseResolver!. By default, a resolver extending the \verb!BaseResolver! will receive the following queries from the \verb!BaseResolver! class: \verb!getOne! and \verb!getAll!.  It will also have the following GraphQL mutations: \verb!create!, \verb!update! and \verb!delete!.  When implementing a resolver that extends a \verb!BaseResolver!, all the developer must provide to the resolver is a service that extends the \verb!BaseService! that will fetch and mutate the data managed by the resolver.

\subsubsection{Connecting the Dots}

The resolver ultimately connects all aspects of the application together.  When a request is received, Nest.js sends the request to the associated resolving function for that request, which will either be defined by the \verb!BaseResolver! for simple CRUD functionality, or will be defined in an implementing resolver for custom functionality.  For all of the requests built into the \verb!BaseResolver!, the function then retrieves the requested data by calling the \verb!ConstructIncludes! function and passing that query into the service which will actually fetch the data from the database.  That resolver then returns that data to the GraphQL server which forwards the result back to the client.

\subsubsection{Custom Functionality}

When the developer extends the \verb!BaseResolver! to create a concrete resolver for a specific type, it is here that they have specific control over what queries and mutations are available on the server.  In this implementing class, the developer can turn on and off creating, updating, and deleting of the object, they can set guards to limit access to certain functions, they can they can implement custom queries, and they can override any functionality defined in the \verb!BaseResolver!.

\subsection{Timestamp Label Example}
Appendix A is an example of the code required to enable CRUD operations for the Timestamp Labels by the GraphQL server for the Practice Liszt app.  In the app, a Timestamp Label defines the name of a bookmark in a video or piece of music.  The main data that has to be collected for a Timestamp Label are the associated collection and the name of the label.  Below I will point out some of the customizations that the framework has enabled and where those features are implemented.

\subsubsection{Data Model/Custom Fields}
In the \verb!TimestampLabelEntity! (Appendix \ref{ex:tl-entity}), the required fields for the Timestamp Label are defined.  Those fields are \verb!timestampCollectionId!, which is used to tie the label to a Timestamp Collection, \verb!title!, and \verb!position!, which allows the user to arrange timestamp labels in the order they occur.  Additionally, a relationship is defined between the Timestamp Label and its collection.  The \verb!title! and \verb!position! fields are marked with \verb!@Field()! decorator, as well, which allows the client to request those specific properties on a GraphQL query.  In this way, the developer has complete control over what data is actually public to the user.  In this example, the timestampCollectionId will never be accessible to the client.

\subsubsection{Creation}
Since the Timestamp Label has to be associated with a Timestamp Collection, the developer has overridden the create function \verb!create! function for the Timestamp Label Service (see Appendix \ref{ex:tl-service}). This is an example of modifying the behavior of creating an object type in the database.  Every time a Timestamp Label is created, the server finds its associated Timestamp Collection, ensures the user has permission to add labels to that Timestamp Collection, and increments the number of labels on that Timestamp Collection.

\subsubsection{Access Permissions}
The Timestamp Label service also limits the Timestamp Collections that a user is allowed to find (see Appendix \ref{ex:tl-service}).  In the \verb!getBaseFindOptions! function, which overrides the \verb!BaseService!, the developer defines a base query that will limit the available Timestamp Labels to either labels on Timestamp Collections that the user created, or Timestamp Collections that are explicitly marked as public.  This restriction is applied to all queries on the database for this object type.

\subsubsection{Modification Permissions}
On a more specific level, in the Timestamp Label resolver, the developer limits the options used to find a Timestamp Label to be updated or deleted (see Appendix \ref{ex:tl-resolver}).  In this situation, the user will only be able to modify Timestamp Labels that they created.  This is more restrictive than the base find options in the service which allow the user to find Timestamp Labels that are in public collections, whereas for modifications, they should not have access to those public collections that they did not create.
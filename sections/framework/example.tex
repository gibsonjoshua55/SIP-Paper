\subsection{Timestamp Label Example}
Appendix A is an example of the code required to enable CRUD operations for the Timestamp Labels by the GraphQL server for the Practice Liszt app.  In the app, a Timestamp Label defines the name of a bookmark in a video or piece of music.  The main data that has to be collected for a Timestamp Label are the associated collection and the name of the label.  Below I will point out some of the customizations that the framework has enabled and where those features are implemented.

\subsubsection{Data Model/Custom Fields}
In the \verb!TimestampLabelEntity! (Appendix \ref{ex:tl-entity}), the required fields for the Timestamp Label are defined.  Those fields are \verb!timestampCollectionId!, which is used to tie the label to a Timestamp Collection, \verb!title!, and \verb!position!, which allows the user to arrange timestamp labels in the order they occur.  Additionally, a relationship is defined between the Timestamp Label and its collection.  The \verb!title! and \verb!position! fields are marked with \verb!@Field()! decorator, as well, which allows the client to request those specific properties on a GraphQL query.  In this way, the developer has complete control over what data is actually public to the user.  In this example, the timestampCollectionId will never be accessible to the client.

\subsubsection{Creation}
Since the Timestamp Label has to be associated with a Timestamp Collection, the developer has overridden the create function \verb!create! function for the Timestamp Label Service (see Appendix \ref{ex:tl-service}). This is an example of modifying the behavior of creating an object type in the database.  Every time a Timestamp Label is created, the server finds its associated Timestamp Collection, ensures the user has permission to add labels to that Timestamp Collection, and increments the number of labels on that Timestamp Collection.

\subsubsection{Access Permissions}
The Timestamp Label service also limits the Timestamp Collections that a user is allowed to find (see Appendix \ref{ex:tl-service}).  In the \verb!getBaseFindOptions! function, which overrides the \verb!BaseService!, the developer defines a base query that will limit the available Timestamp Labels to either labels on Timestamp Collections that the user created, or Timestamp Collections that are explicitly marked as public.  This restriction is applied to all queries on the database for this object type.

\subsubsection{Modification Permissions}
On a more specific level, in the Timestamp Label resolver, the developer limits the options used to find a Timestamp Label to be updated or deleted (see Appendix \ref{ex:tl-resolver}).  In this situation, the user will only be able to modify Timestamp Labels that they created.  This is more restrictive than the base find options in the service which allow the user to find Timestamp Labels that are in public collections, whereas for modifications, they should not have access to those public collections that they did not create.
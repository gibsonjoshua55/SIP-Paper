\subsection{BaseEntity}

To avoid duplicating logic around retrieving data out the database, it is helpful that all database entities share a common structure.  When I began working at Maestro, I was introduced to this pattern through the software team's use of a parent class called \verb!BaseEntity!.  This data model includes fields such as \verb!id!, \verb!identity! (a Universally Unique Identifier or UUID, instead of a integer id), and \verb!createdAt!, \verb!updatedAt!, and \verb!deletedAt! timestamps.  Once this class has been established all other data models in the application should extend this \verb!BaseEntity!.  This way, for any type of data in the database, I know how to retrieve the records since they all have an id, and identity. I also know that when I create those records, consistent metadata will be kept regarding their creation, modification, and deletion. When creating a service that will handle these operation, this baseline knowledge about each is crucial.

By taking a subclassing approach to the data model, duplicating logic around these duplicated fields can also be avoided.  Looking back to the code example in Appendix \ref{ex:tl-entity}, the code does not specify that a \verb!TimestampLabel! has an \verb!id!, \verb!identity!, etc. Those basic fields are inherited from \verb!CreatedByEntity! which extends \verb!BaseEntity!.  This leads to yet another benefit of subclassing: there can be multiple levels of inheritance. For example, for many tables in the database, the user that creates the record should be tracked as well.  Having a \verb!CreatedByEntity! that extends \verb!BaseEntity! means that the \verb!created_by_id!, \verb!updated_by_id!, and \verb!deleted_by_id! can be added to this one model and never duplicated again throughout the code base.  Other models that are created by a user simply extend the \verb!CreatedByEntity!, thus receiving all \verb!BaseEntity! and \verb!CreatedByEntity! fields.